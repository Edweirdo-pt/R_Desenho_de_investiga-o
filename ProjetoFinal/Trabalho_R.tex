% Options for packages loaded elsewhere
\PassOptionsToPackage{unicode}{hyperref}
\PassOptionsToPackage{hyphens}{url}
\documentclass[
]{article}
\usepackage{xcolor}
\usepackage[margin=1in]{geometry}
\usepackage{amsmath,amssymb}
\setcounter{secnumdepth}{-\maxdimen} % remove section numbering
\usepackage{iftex}
\ifPDFTeX
  \usepackage[T1]{fontenc}
  \usepackage[utf8]{inputenc}
  \usepackage{textcomp} % provide euro and other symbols
\else % if luatex or xetex
  \usepackage{unicode-math} % this also loads fontspec
  \defaultfontfeatures{Scale=MatchLowercase}
  \defaultfontfeatures[\rmfamily]{Ligatures=TeX,Scale=1}
\fi
\usepackage{lmodern}
\ifPDFTeX\else
  % xetex/luatex font selection
\fi
% Use upquote if available, for straight quotes in verbatim environments
\IfFileExists{upquote.sty}{\usepackage{upquote}}{}
\IfFileExists{microtype.sty}{% use microtype if available
  \usepackage[]{microtype}
  \UseMicrotypeSet[protrusion]{basicmath} % disable protrusion for tt fonts
}{}
\makeatletter
\@ifundefined{KOMAClassName}{% if non-KOMA class
  \IfFileExists{parskip.sty}{%
    \usepackage{parskip}
  }{% else
    \setlength{\parindent}{0pt}
    \setlength{\parskip}{6pt plus 2pt minus 1pt}}
}{% if KOMA class
  \KOMAoptions{parskip=half}}
\makeatother
\usepackage{longtable,booktabs,array}
\usepackage{calc} % for calculating minipage widths
% Correct order of tables after \paragraph or \subparagraph
\usepackage{etoolbox}
\makeatletter
\patchcmd\longtable{\par}{\if@noskipsec\mbox{}\fi\par}{}{}
\makeatother
% Allow footnotes in longtable head/foot
\IfFileExists{footnotehyper.sty}{\usepackage{footnotehyper}}{\usepackage{footnote}}
\makesavenoteenv{longtable}
\usepackage{graphicx}
\makeatletter
\newsavebox\pandoc@box
\newcommand*\pandocbounded[1]{% scales image to fit in text height/width
  \sbox\pandoc@box{#1}%
  \Gscale@div\@tempa{\textheight}{\dimexpr\ht\pandoc@box+\dp\pandoc@box\relax}%
  \Gscale@div\@tempb{\linewidth}{\wd\pandoc@box}%
  \ifdim\@tempb\p@<\@tempa\p@\let\@tempa\@tempb\fi% select the smaller of both
  \ifdim\@tempa\p@<\p@\scalebox{\@tempa}{\usebox\pandoc@box}%
  \else\usebox{\pandoc@box}%
  \fi%
}
% Set default figure placement to htbp
\def\fps@figure{htbp}
\makeatother
\setlength{\emergencystretch}{3em} % prevent overfull lines
\providecommand{\tightlist}{%
  \setlength{\itemsep}{0pt}\setlength{\parskip}{0pt}}
\usepackage{bookmark}
\IfFileExists{xurl.sty}{\usepackage{xurl}}{} % add URL line breaks if available
\urlstyle{same}
\hypersetup{
  pdftitle={Trabalho\_R},
  hidelinks,
  pdfcreator={LaTeX via pandoc}}

\title{Trabalho\_R}
\author{}
\date{\vspace{-2.5em}}

\begin{document}
\maketitle

\section{Pergunta 1 -- Classificação das
Variáveis}\label{pergunta-1-classificauxe7uxe3o-das-variuxe1veis}

Nesta etapa, procedeu-se à classificação técnica de todas as variáveis
presentes na base de dados, identificando a sua natureza e a respetiva
escala de medição. Esta classificação é fundamental para selecionar os
métodos estatísticos e as representações gráficas mais adequadas.

\begin{longtable}[]{@{}lll@{}}
\toprule\noalign{}
Variável & Tipo de Variável & Escala de Medição \\
\midrule\noalign{}
\endhead
\bottomrule\noalign{}
\endlastfoot
\textbf{Name} & Qualitativa Nominal & Nominal \\
\textbf{Age} & Quantitativa Contínua & Razão \\
\textbf{Gender} & Qualitativa Nominal & Nominal \\
\textbf{Blood Type} & Qualitativa Nominal & Nominal \\
\textbf{Medical Condition} & Qualitativa Nominal & Nominal \\
\textbf{Date of Admission} & Quantitativa Contínua & Intervalo \\
\textbf{Doctor / Hospital} & Qualitativa Nominal & Nominal \\
\textbf{Insurance Provider} & Qualitativa Nominal & Nominal \\
\textbf{Billing Amount} & Quantitativa Contínua & Razão \\
\textbf{Room Number} & Qualitativa Nominal & Nominal \\
\textbf{Admission Type} & Qualitativa Nominal & Nominal \\
\textbf{Discharge Date} & Quantitativa Contínua & Intervalo \\
\textbf{Medication} & Qualitativa Nominal & Nominal \\
\textbf{Test Results} & Qualitativa Nominal & Nominal \\
\end{longtable}

\section{Pergunta 2 -Caracterização de
variáveis}\label{pergunta-2--caracterizauxe7uxe3o-de-variuxe1veis}

Vamos usar o R Base para os gráficos da \texttt{Age} (Histograma e
Boxplot) para evitar que o R tente instalar o \texttt{patchwork} ou o
\texttt{ggplot2} a meio do PDF, o que ``suja'' o documento.

\section{2.1 ``Age''}\label{age}

\textbf{Classificação:} Quantitativa Contínua (Escala de Razão).

\textbf{Justificação:} Por ser uma variável quantitativa, utilizam-se
medidas de tendência central e dispersão. O histograma permite observar
a forma da distribuição, enquanto o boxplot identifica a existência de
valores atípicos.

\begin{verbatim}
##    Min. 1st Qu.  Median    Mean 3rd Qu.    Max. 
##   18.00   34.00   51.00   51.25   68.00   85.00
\end{verbatim}

\begin{verbatim}
## [1] 19.67971
\end{verbatim}

\pandocbounded{\includegraphics[keepaspectratio]{Trabalho_R_files/figure-latex/unnamed-chunk-1-1.pdf}}
\textbf{Exploração} A idade média dos pacientes na amostra é de 51.25
anos. A proximidade entre a média e a mediana, aliada à observação do
histograma, sugere uma distribuição relativamente uniforme/simétrica e o
boxplot confirma a inexistência de outliers, indicando uma amostra sem
casos extremos que influenciem a média.

\section{2.2 ``Gender''}\label{gender}

\textbf{Classificação:} Qualitativa Nominal(dicotómica)

\textbf{Justificação:} Por tratar-se de uma variável com apenas duas
categorias possíveis (Masculino e Feminino), a caracterização é feita
através de frequências absolutas e relativas. O gráfico de barras é a
representação ideal para comparar a proporção entre os dois grupos,
permitindo verificar visualmente se a amostra é equilibrada.

\pandocbounded{\includegraphics[keepaspectratio]{Trabalho_R_files/figure-latex/unnamed-chunk-2-1.pdf}}

\textbf{Exploração} A variável apresenta apenas duas categorias
possíveis: Female (Feminino) e Male (Masculino). Observa-se que a
amostra é composta por 2735 mulheres e 2696 homens, correspondendo a
50.4\% e 49.6\%, respetivamente. Esta distribuição quase paritária
(muito próxima dos 50\% para cada categoria) indica que a amostra está
bem equilibrada em termos de género, não havendo predominância de um
sexo sobre o outro.

\section{2.3 ``Blood.Type''}\label{blood.type}

\textbf{Classificação:} Variável qualitativa nominal com 8 categorias.

\textbf{Justificação:} Utilizou-se um gráfico de barras ordenado de
forma decrescente, o que facilita a identificação imediata da ``Moda''
(o grupo mais frequente).

\pandocbounded{\includegraphics[keepaspectratio]{Trabalho_R_files/figure-latex/unnamed-chunk-3-1.pdf}}

A análise da distribuição dos grupos sanguíneos revela que o tipo O+ é o
mais frequente na amostra, registando-se em 698 pacientes
\texttt{(r\ round(blood\_perc{[}1{]},\ 1)\%)}. Seguem-se os grupos A+ e
AB- . Esta diversidade é característica de bases de dados hospitalares
amplas, não se verificando uma predominância absoluta que comprometa a
representatividade de um grupo específico.

\section{2.4 ``Medical.Condition''}\label{medical.condition}

\textbf{Classificação:} Qualitativa Nominal

\textbf{Justificação:}Para a variável Medical.Condition, optou-se por um
gráfico de barras horizontal. Esta escolha permite uma leitura clara das
patologias, evitando a sobreposição de texto, e facilita a comparação da
carga de doença entre as diferentes categorias.

\pandocbounded{\includegraphics[keepaspectratio]{Trabalho_R_files/figure-latex/unnamed-chunk-4-1.pdf}}

\end{document}
